\documentclass[12pt,a4paper]{article}
%% margin
\usepackage{geometry}
\geometry{top=1.5cm,bottom=1.5cm,left=1cm,right=1cm}
%% color
\usepackage{xcolor}
%% math
\usepackage{amsmath}
\usepackage{amssymb}
\usepackage{mathtools}
\usepackage{mathpazo}
%% multi-column
\usepackage{multicol}
%% set sans serif and typewriter font to Khmer OS
\usepackage[no-math]{fontspec}
\XeTeXlinebreaklocale "kh"
\setmainfont[Ligatures=TeX]{Khmer OS Content}
\setsansfont[Ligatures=TeX]{Khmer OS Content}
\setmonofont[Ligatures=TeX]{Khmer OS Content}
%% redefine enumerate label
\renewcommand{\theenumi}{\Roman{enumi}}
\renewcommand{\theenumii}{\arabic{enumii}}
\renewcommand{\theenumiii}{\alph{enumiii}}
\renewcommand{\labelenumi}{\textcolor{blue}{\bfseries\theenumi.}}
\renewcommand{\labelenumii}{\textcolor{blue}{\bfseries\theenumii.}}
\renewcommand{\labelenumiii}{\textcolor{blue}{\bfseries\theenumiii.}}
%% redefine itemize label
\renewcommand{\labelitemi}{\textcolor{blue}{\textbullet}}
\renewcommand{\labelitemii}{\textcolor{blue}{\ensuremath{\circ}}}
\renewcommand{\labelitemiii}{\textcolor{blue}{\textasteriskcentered}}
%% horizontal and vertical paragraph alignment
\raggedright
\raggedbottom
%%
\begin{document}
\begin{enumerate}
	\item (១៥ ពិន្ទុ) គេមានចំនួនកុំផ្លិច $ z=(\sqrt{6}+\sqrt{2})+i(\sqrt{6}-\sqrt{2}) $~។
	\begin{enumerate}
		\item សរសេរ $ z^2 $ ជាទម្រង់ត្រីកោណមាត្ររួចទាញរកទម្រង់ត្រីកោណមាត្រនៃ $ z $~។
		\item រកចំនួនគត់ $ n $ វិជ្ជមានតូចបំផុតដែល $ z^n $ ជាចំនួនពិត។
	\end{enumerate}
	\item (១៥ ពិន្ទុ) គណនាលីមីតខាងក្រោម៖
	\begin{multicols}{3}
		\begin{enumerate}
			\item $ \lim\limits_{x\to 3}\dfrac{x^2-4x+3}{9-x^2} $
			\item $ \lim\limits_{x\to 2}\dfrac{x\sqrt{x}-2\sqrt{2}}{\sqrt{x}-\sqrt{2}} $
			\item $ \lim\limits_{x\to 0}\dfrac{e^{x^2}+\sin (x^2)-1}{2x\sin x} $
		\end{enumerate}
	\end{multicols}
	\item (១៥ ពិន្ទុ) គេមានអនុគមន៍ $ f(x)=\dfrac{x^2+3x-7}{(x+2)(x-1)^2} $~។
	\begin{enumerate}
		\item កំណត់ចំនួនពិត $ a,b,c $ ដែល $ f(x)=\dfrac{a}{x-1}+\dfrac{b}{(x-1)^2}+\dfrac{c}{x+2} $
		\item គណនាអាំងតេក្រាល $ \displaystyle\int f(x)\mathrm{d}x $~។
	\end{enumerate}
	\item (១៥ ពិន្ទុ) នៅក្នុងកន្ត្រកមួយមានពងទា``កូន''ចំនួន $ 5 $ គ្រាប់ ពងទា``សាប''ចំនួន $ 7 $ គ្រាប់ និងពងទា``ខូច''ចំនួន $ 3 $ គ្រាប់។ ក្មេងម្នាក់ចាប់យកពងទា $ 5 $ គ្រាប់ ដោយចៃដន្យពីក្នុងកន្ត្រកនោះ។\\
	គណនាប្រូបាបនៃព្រឹត្តិការណ៍៖
	\begin{enumerate}
		\item $ A: $ ``បានពងទាកូន $ 2 $ គ្រាប់ ពងទាសាប $ 2 $ គ្រាប់ និងពងទាខូច $ 1 $ គ្រាប់''
		\item $ B: $ ``បានពងទាកូន $ 4 $ គ្រាប់''
		\item $ C: $ ``បានពងទាខូចយ៉ាងតិច $ 1 $ គ្រាប់''
	\end{enumerate}
	\item (៣៥ ពិន្ទុ) 
	\begin{itemize}
		\item[ផ្នែក A.)] គេមានអនុគមន៍ $ g $ កំណត់លើ $ (0,+\infty) $ ដោយ $ g(x)=x^2+1-\ln x $~។
		\begin{enumerate}
			\item គណនាដេរីវេ $ g'(x) $ នៃអនុគមន៍ $ g(x) $ រួចទាញរកអថេរភាពនៃ $ g $~។
			\item គូសតារាងអថេរភាពនៃ $ g $ ហើយទាញរកសញ្ញានៃ $ g $~។
		\end{enumerate}
		\item[ផ្នែក B.)] គេមាន $ f $ ជាអនុគមន៍កំណត់លើ $ (0,+\infty) $ ដោយ $ f(x)=1-x-\dfrac{\ln x}{x} $ ហើយមានក្រាប $ C $~។
		\begin{enumerate}
			\item គណនា $ f'(x) $ ហើយទាញ $ f'(x) $ ជាអនុគមន៍នៃ $ g(x) $ ព្រមទាំងបញ្ជាក់សញ្ញា $ f'(x) $ លើ $ (0,+\infty) $~។
			\item គណនាលីមីត​នៃអនុគម៍ $ f $ ត្រង់ $ 0^+ $ និង $ +\infty $ រួចគូសតារាងអថេរភាពនៃ $ f $~។
			\item បង្ហាញថាបន្ទាត់ $ d:\;y=-x+1 $ ជាអាស៊ីមតូតទ្រេតនៃក្រាប $ C $ ខាងមែក $ +\infty $~។\\
			រួចសិក្សាទីតាំងរវាងក្រាប $ C $ និងបន្ទាត់ $ d $~។
			\item គូសខ្សែកោង $ C $ និងបន្ទាត់ $ d $ ក្នុងតម្រុយតែមួយ។
		\end{enumerate}
	\end{itemize}
	\item (៣០ ពិន្ទុ) 
	\begin{itemize}
		\item[ផ្នែក A.)] គេឲ្យសមីការទូទៅនៃអេលីប $ E:\;9x^2+25y^2=225 $~។
		\begin{enumerate}
			\item រកប្រវែងអ័ក្សធំ ប្រវែងអ័ក្សតូច និងកូអរដោនេកំពូលទាំងពីរ។
			\item សង់អេលីប​ $ E $~។
		\end{enumerate}
		\item[ផ្នែក B.)] ក្នុងលំហរប្រដាប់ដោយតម្រុយអតូណរម៉ាល់មានទិសដៅវិជ្ជមាន $ (O,\vec{i},\vec{j},\vec{k}) $ គេមានចំណុច​\\
		$ A(1,0,1),B(2,1,2),C(2,3,1) $ និង $ D(1,2,3) $~។
		\begin{enumerate}
			\item សរសេរវ៉ិចទ័រ $ \overrightarrow{AB},\overrightarrow{AC},\overrightarrow{AD} $ រួចគណនា $ \overrightarrow{AB}\times\overrightarrow{AC} $ និង $ (\overrightarrow{AB}\times\overrightarrow{AC})\cdot\overrightarrow{AD} $
			\item សរសេរសមីការទូទៅនៃប្លង់ $ ABC $ ហើយបង្ហាញថា $ D $ មិនមែនជាចំណុចនៃប្លង់ $ ABC $
			\item សរសេរសមីការឆ្លុះនៃបន្ទាត់ $ L $ កាត់តាម $ D $ ហើយកែងនឹងប្លង់ $ ABC $~។
		\end{enumerate}
	\end{itemize}
\end{enumerate}
\end{document}